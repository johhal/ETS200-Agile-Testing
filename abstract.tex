This paper aims to conclude whether usage of Test-Driven Development reflects in the quality of the final product or not. This report is a summary of several articles about Test-Driven development along with a conclusion in the end trying to decide the effect of Test-Driven Development. It will also be discussed how the return on investment of TDD depends on the slower development speed compared to the higher code quality of TDD.  The necessity for a mature development team as well as experienced developers will be discussed as well. 
Finally, this paper concludes that the increased production cost that comes with TDD will pay off in increased code quality, thus reducing the cost of maintaining the product. 