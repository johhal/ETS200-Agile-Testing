This paper aims to conclude whether usage of Test-Driven Development reflects in the quality of the final product or not. This report is a summary of several articles about Test-Driven development along with a conclusion in the end trying to decide the effect of Test-Driven Development. It will also be discussed how the return on investment of TDD depends on the slower development speed compared to the higher code quality of TDD.  The necessity for a mature development team as well as experienced developers will be discussed as well. 
This paper concludes that the increased production cost that comes with TDD will pay off 

drawback of using TDD is the increased production cost that comes with the extra work
performed by the developers. The extra time spent on writing test cases will however pay of
if quite enough bugs and defects are avoided during development. The final quality of the
product increases with the decreasing number of bugs and defects, which also dramatically
decreases the cost of maintaining the product with bug fixes later on.

