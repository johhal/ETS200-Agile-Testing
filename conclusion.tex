The drawback of using TDD is the increased production cost that comes with the extra work performed by the developers. The extra time spent on writing test cases will however pay of if quite enough bugs and defects are avoided during development. The final quality of the product increases with the decreasing number of bugs and defects, which also dramatically decreases the cost of maintaining the product with bug fixes later on.

Developers commitment to TDD is imperative to successively use TDD in a software project. If merely one developer ignores the TDD practices the foundation in TDD, namely the code correctness verification, will malfunction since the entire code functionality will not be covered by tests.

To successfully implement TDD in a software engineering process the maturity of the developers has been proven important; both in accepting the usage of TDD in place of Test-Last and in the development effectiveness provided by TDD. Less experienced developers do not perform as well as experienced do in the beginning when using TDD but will, however, after a while learn and the results will improve.
  	
TDD provides a team with means to coordinate their work and make sure that after changes has been done the old functionality still works. It also provides a ``safety-net'' for refactorisation of the code which will lead to conformation of the code to common coding standards as well as decreasing the complexity of the code, which will greatly help the developers to communicate and collaborate through the design and layout of the code.
	  	
TDD isn't a process in software engineering but rather a method, a tool, which is applicable to more or less all agile development processes, and varieties of TDD could possibly be used in some non-agile development processes as well. Further, if TDD is implemented correctly in an experienced and devoted team it will most likely increase the final quality of the product.  