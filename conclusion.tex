The drawback of using TDD is the increased production cost that comes with the extra work performed by the developers.
The extra time spent on writing testcases will pay of since much less bugs will be found in the software after an release using TDD, dramaticly decreasing the cost of maintaining the product with bug fixes.

The commitment to TDD by the developers is imperative for the success of TDD in a project since if just one developer
is not onboard with it and does not follow the policies around TDD or write tests it will cause other developers problem since they
trust the test to say if the system is working as expected or not. The maturity of developers have been proved important in accepting TDD
in a team and they also perform best with TDD. Less experienced developers does not perform as well but will after using TDD progress and
the results will improve.
TDD provides a team with means to coordinate their work and make sure that after changes has been done they dont interfere with old functionality, but also provides safety for refactorisation of the code, which in the case of new simpler design, conforming the code to common code standards and decreasing the complexity of the code greatly help the developers to communicate and collaborate through the design and layout of the code. 

