%Introduction: introduce the chosen area. The purpose is to give an introduction to the reader who is not
%familiar to the specific area, but knows software engineering and testing generally.

%This section will introduce the reader to the concept of Test Driven Development in Agile Processes and explain how it is performed and how it differs from other software development processes.

This report concerns the area of Test-first Development in Agile Processes and aims to describe and analyse how the final quality of a system is affected by this development method. 

Despite being an old software development method, Test-first Development, also known as Test Driven Development, TDD, has become more and more popular during the last years, inter alia, due to being a central practise of Extreme Programming, XP~\cite{georgeandwilliams}. In Test-Driven development the developers write unit tests for the functionality they are implementing before they write the actual code~\cite{beckTestDriven}. The method is related to the Test-First practices used in XP, ~\cite{beckXP}, but today it is used with other development methods as well~\cite{MSNET}. 

As described by M\"{u}ller and Hagner in~\cite{mullerandhagner}, TDD urges the developers to:
\begin{itemize}
\item	develop programs more capable of accepting changes
\item 	program faster
\item	increase confidence of both developer and customer by seeing all tests pass as well as fail
\item 	reduce defect rates 
\item	understand the program better
\end{itemize}

\noindent Further, M\"{u}ller and Hagner, ~\cite{mullerandhagner}, describes that TDD aims to:
\begin{itemize}
\item 	ensure that programmed features cannot be lost
\item 	force developers to think about and write unit tests 
\item	automate test execution
\item	prevent a program from regressing due to ongoing testing
\item 	provide a test-suite as basis for refactoring
\end{itemize}

\noindent The TDD process is often compared to more traditional Test-last processes, in terms of development time, resultant code quality and understandability~\cite{georgeandwilliams}, or in terms of programming efficiency, resultant code reliability and program understanding~\cite{mullerandhagner}. 

This project will focus on how the quality measure, as mentioned above, is defined and how it differs amongst the different types of software development processes as well as how it is affected by TDD in Agile Processes. 