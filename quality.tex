From a software quality assurance perspective, software is defined  as a ``combination of computer program (the `code'), procedures, documentation, and data necessary for operating the software''\cite{Galin}. Software quality, on the other hand, defines ``the degree of conformance to specific functional requirements, specific software quality and Good Software Engineering Practices (GSEP)'', as stated by Pressman~\cite{Pressman}. 

This report uses code quality as a term for both ``Software functional quality'' and ``Software structural quality'' or ``Non-functional quality'' as Lawrence and Julio,~\cite{Chung}, describe it. The former terms refer more to code per se and the latter more to the behaviour of the code such as software affection or usability. 

Code quality can be defined and categorised in numerous ways and it has been standardised differently in the past. There are international standards for product quality, ISO-9126 and ISO-25000,~\cite{ISO9126}, which define a basic quality model. This report uses these standards to define Code quality but also the global standard made by the ``Consortium for IT Software Quality'' about business value and customer satisfaction,~\cite{cisq}, which is based on the standards mentioned above. 

ISO-9126 is an international standard for product quality and defines six major desirable characteristics that should be measured in order to guarantee good code quality. Each characteristic is divided into sub-characteristic and further split into attributes which this report do not aim to explain. A good conclusion of the interpretation with 
``Software quality is the degree to which software possesses a desirable combination of attribute (e.g., reliability, interoperability)''~\cite{ISO1061}. 

The six major characteristics put forth in the standard are, as seen in Figure~\ref{fig:isoPic}: 
\begin{itemize}
\item Functionality 
\item Reliability
\item Usability
\item Efficiency
\item Maintainability
\item Portability
\end{itemize}

We use the term "code quality" in this paper in the same way as product quality is mentioned above but overlooking all other things but the code. That is we do not take in consideration the Usability of the user manual or the Maintainability of the documentation. 
