
TDD was used in the 1960s in the NASA Project Mercury,~\cite{NASA}, and was then rediscovered by Kent Beck (founder of XP) in 2003. Beck claims that it encourages simple design and inspires confidence~\cite{beckXP}. In XP the developers break down the functionality to be implemented into smaller problems, also known as stories. When using TDD along with XP the work flow follows the illustration in Figure~\ref{fig:tddPic} in Appendix. 

After a few test cases are developed the code to make them pass is implemented. Generally the test cases wont even compile before the code is implemented. If the test cases pass, the developer either writes more test cases for the next piece of functionality or moves on to the next story depending on whether the previous story is finished or not. If the test cases fail the developer has to go back and rework the newly implemented code.


By focusing on writing only code necessary to pass tests, it is easier to keep a clean and clear design compared to when using other methods~\cite{beckXP}. When used in XP projects, TDD helps to full fill the ``Simple Design'' practice which basically is keeping the code as simple as possible and only implement functionality that is needed at the moment. By doing this the code is easier to work with and it simplifies future integrations. 

When performing TDD, developers usually use some sort of testing tool. An example of a testing tool is JUnit which can be used when developing in Java. JUnit includes a set of assert statements which can be used to verify the output given when unit testing. It also provides Before- and After statements which helps the developer to set up and tear down variables that are needed for several tests. With this functionality, test tools helps the developer to manage the vast amount of test cases that are produced.

There are several claims that TDD increases both code quality and productivity~\cite{beckXP}~\cite{erdogmus}. For instance, they argue that the test cases are more efficient since the tests have been developed from requirements instead of existing code. However, there are others who showed in studies that there is no difference in, or even a decrease in, the overall results~\cite{tddInvest}~\cite{mullerandhagner}. Section~\ref{analysis} gives a summary of the pros and cons with TDD.