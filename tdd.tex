In Test-Driven development the developers writes unit tests for the functionality they are implementing before they write the actual code~\cite{beckTestDriven}. The method is related to the Test-First practices used in eXtreme-Programming (XP)~\cite{beckXP} but today it is used with other development methods as well~\cite{MSNET}. TDD was used in the 1960’s in the NASA Project Mercury~\cite{NASA} and then rediscovered by Kent Beck (who created XP) in 2003 who claims that it encourages simple design and inspires confidence~\cite{beckXP}. In XP the developers break down the functionality to be implemented into smaller problems, also known as stories. When using TDD along with XP the workflow is according to the scheme in Figure~\ref{fig:tddPic}. 

After a few test cases are developed the code to make them pass are implemented. Generally the test cases wont even compile before the code is implemented. If the test cases pass, the developer either writes more test cases for the next piece of functionality or moves on to the next story depending on whether the previous story is finished or not. If the test cases fail the developer has to go back and rework the newly implemented code.


By focusing on writing only code necessary to pass tests, it is easier to keep a clean and clear design compared to when using other methods~\cite{beckXP}. When used in XP projects TDD helps to full fill the “Simple Design” practice which basically is keeping the code as simple as possible and only implement functionality that is needed at the moment. By doing this the code is easier to work with and it simplifies future integrations. 


When performing TDD, developers usually use some sort of testing tool. An example of a testing tool is JUnit which can be used when developing in Java. JUnit includes a set of assert statements which can be used to verify the output given when unit testing. It also provides Before- and After statements which helps the developer to set up and tear down variables that are needed for several tests. With this functionality, test tools helps the developer to manage the vast amount of test cases that are produced.


There are several claims that TDD increases both code quality and productivity~\cite{beckXP}~\cite{erdogmus}. For instance they argue that the test cases are more efficient since the tests have been developed from requirements instead of existing code. However there are others who showed studies that there is no difference in, or even a decrease in, the overall results~\cite{tddInvest}~\cite{mullerandhagner}. See Section~\ref{analysis} for a summary of the pros and cons with TDD.